\documentclass{article}
\usepackage{amsmath,amsthm,amsfonts}

\newtheorem{definition}{Definition}[section]
\newtheorem{theorem}{Theorem}[section]
\newtheorem{corollary}{Corollary}[theorem]
\newtheorem{lemma}[theorem]{Lemma}

% generated from chatgpt gpt4: https://chat.openai.com/share/c4e54d91-eec3-4fd5-ba69-3ce4afb049fc

\begin{document}

%% --------------------
\section{Vectors and Matrices}
In this section, we introduce the fundamental objects in linear algebra: vectors and matrices.
We also consider one of the basic operations with matrices: matrix addition.

\begin{definition}[Vector]
A vector is a mathematical object that has magnitude and direction, and which adheres to the laws of addition and scalar multiplication.
\end{definition}

\begin{definition}[Matrix]
A matrix is a rectangular array of numbers arranged in rows and columns.
\end{definition}

\begin{theorem}[Matrix Addition]
The sum of two matrices A and B, each of size m x n, is another matrix C = A + B, also of size m x n.
\end{theorem}

\begin{proof}
If A = [a_{ij}] and B = [b_{ij}], then their sum C = [c_{ij}] is given by c_{ij} = a_{ij} + b_{ij} for each i and j.
\end{proof}

\end{document}
